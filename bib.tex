\addcontentsline{toc}{part}{Книги}
\begin{thebibliography}{99}

\secly{Разработка языков программирования}

\bibitem{dragon} \bibfig{bib/dragon.jpg}\ \emph{Dragon
Book}:
\textbf{Компиляторы. Принципы, технологии, инструменты.}\\
Альфред Ахо, Рави Сети, Джеффри Ульман.\\
Издательство Вильямс, 2003.\\ ISBN 5-8459-0189-8

\bibitem{dragonen} \textbf{Compilers: Principles, Techniques, and Tools}\\
Aho, Sethi, Ullman\\Addison-Wesley, 1986.\\ISBN 0-201-10088-6

\bibitem{sicp} \textbf{\emph{SICP}: Структура и интерпретация компьютерных
программ}\\
ISBN 5-98227-191-8

\bibitem{fild}
\href{https://drive.google.com/file/d/0B0u4WeMjO894b2UxcmVvdi01TEU/view?usp=sharing}{\textbf{Функциональное
программирование}}\\
Филд А., Харрисон П.\\
М.: Мир, 1993\\ISBN 5-03-001870-0

\subsecly{ML}

\bibitem{fourman}
\url{http://homepages.inf.ed.ac.uk/mfourman/teaching/mlCourse/notes/L01.pdf}\\
\textbf{Basics of Standard ML}\\
\copyright\ Michael P. Fourman\\
перевод \ref{fourmanru}

\bibitem{gentleml}
\url{http://www.soc.napier.ac.uk/course-notes/sml/manual.html}\\
\textbf{A Gentle Introduction to ML}\\
\copyright\ Andrew Cumming, Computer Studies, Napier University, Edinburgh

\bibitem{harper} \url{http://www.cs.cmu.edu/~rwh/smlbook/book.pdf}\\
\textbf{Programming in Standard ML}\\
\copyright\ Robert Harper, Carnegie Mellon University

\end{thebibliography}
