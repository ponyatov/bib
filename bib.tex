\addcontentsline{toc}{part}{Книги}

\secly{Книги must have любому техническому специалисту}

\subsecly{Математика, физика, химия}

\begin{itemize}[nosep]
  \item Бермант \textbf{Математический анализ}\ \cite{bermant}
  \item Кремер \textbf{Теория вероятностей и матстатистика}\ \cite{kremer}
  \item Ван дер Варден \textbf{Математическая статистика}\ \cite{varver}
  \item Смит \textbf{Цифровая обработка сигналов}\ \cite{smitdsp}
  \item Тихонов, Самарский \textbf{Математическая физика}\ \cite{ts,tszad}
  \item Кострикин \textbf{Введение в алгебру}\ \cite{kostalg,kostlin}
  \item Ван дер Варден \textbf{Алгебра}\ \cite{vanalg}
\end{itemize}

\paragraph{Фейнмановские лекции по физике}
\begin{enumerate}[nosep]
  \item Современная наука о природе. Законы механики.\ \cite{fein1}
  \item Пространство. Время. Движение.\ \cite{fein2}
  \item Излучение. Волны. Кванты.\ \cite{fein3}
  \item Кинетика. Теплота. Звук.\ \cite{fein4}
  \item Электричество и магнетизм\ \cite{fein5}
  \item Электродинамика.\ \cite{fein6}
  \item Физика сплошных сред.\ \cite{fein7}
  \item Квантовая механика 1.\ \cite{fein8}
  \item Квантовая механика 2.\ \cite{fein9}
\end{enumerate}
\bigskip

\begin{itemize}[nosep]
  \item Цирельсон \textbf{Квантовая химия}\ \cite{cirel}
  \item Розенброк \textbf{Вычислительные методы для инженеров-химиков}\
  \cite{calchem}
  \item Шрайвер Эткинс \textbf{Неорганическая химия}\ \cite{shriver}
\end{itemize} 

\subsecly{Обработка экспериментальных данных и метрология}

\begin{itemize}[nosep]
  \item Князев, Черкасский \textbf{Начала обработки экспериментальных данных}
  \cite{data}
\end{itemize}

\subsecly{Программирование}

\begin{itemize}[nosep]
  \item \textbf{Система контроля версий \prog{Git}\ и git-хостинга
  GitHub}\\хранение наработок с полной историей редактирования, правок, релизов
  для разных заказчиков или вариантов использования
  \item \textbf{Язык Python}\ \cite{py}\\ написание простых скриптов обработки
  данных, автоматизации, графических оболочек и т.п. утилит
  \item \textbf{Язык \cpp, утилиты GNU toolchain} \cite{bogo,kumar}\ (gcc/g++,
  make, ld)\\базовый Си, ООП очень кратко\note{наследование, полиморфизм,
  операторы для пользовательских типов, использование библиотеки STL}, без
  излишеств профессионального программирования\note{мегабиблиотека Boost,
  написание своих библиотек шаблонов и т.п.}, чисто вспомогательная роль для
  написания вычислительных блоков и критичных к скорости/памяти секций,
  использовать в связке с Python.\\
  Знание базового Си \emph{критично при использовании микроконтроллеров},
  из \cpp\ необходимо владение особенностями использования ООП и управления
  крайне ограниченной памятью: пользовательские менеджеры памяти, статические
  классы.
  \item Использование утилит \textbf{flex/bison}\\обработка текстовых форматов
  данных, часто необходимая вещь.
\end{itemize}

\begin{thebibliography}{99}

\secly{Разработка языков программирования и компиляторов}

\bibitem{dragon} \bibfig{../bib/dragon.jpg}\ \emph{Dragon Book}\\
\textbf{Компиляторы. Принципы, технологии, инструменты.}\\
Альфред Ахо, Рави Сети, Джеффри Ульман.\\
Издательство Вильямс, 2003.\\ ISBN 5-8459-0189-8

\bibitem{dragonen} \textbf{Compilers: Principles, Techniques, and Tools}\\
Aho, Sethi, Ullman\\Addison-Wesley, 1986.\\ISBN 0-201-10088-6

\bibitem{sicp} \bibfig{../bib/sicp.jpg}\ \textbf{\emph{SICP}\\
\href{https://drive.google.com/file/d/0B0u4WeMjO894X3lnWmhjUktKRk0/view?usp=sharing}{Структура
и интерпретация компьютерных программ}}\\
Харольд Абельсон, Джеральд Сассман\\
ISBN 5-98227-191-8\\
EN: \url{web.mit.edu/alexmv/6.037/sicp.pdf}

\bibitem{fild} \bibfig{../bib/fild.jpg}\\
\href{https://drive.google.com/file/d/0B0u4WeMjO894b2UxcmVvdi01TEU/view?usp=sharing}{\textbf{Функциональное
программирование}}\\
Филд А., Харрисон П.\\
М.: Мир, 1993\\ISBN 5-03-001870-0

\bibitem{henderson} \bibfig{../bib/henderson.jpg}\\
\href{https://drive.google.com/file/d/0B0u4WeMjO894VHhRb0tfTWNWV1k/view?usp=sharing}{Функциональное
программирование: применение и реализация}\\
П.Хендерсон\\
М.: Мир, 1983

\bibitem{llvm} \bibfig{../bib/llvm.jpg}
\textbf{LLVM. Инфраструктура для разработки компиляторов}\\
Бруно Кардос Лопес, Рафаэль Аулер

\subsecly{Lisp/Sheme}

\subsecly{Haskell}

\subsecly{ML}

\bibitem{fourman}
\url{http://homepages.inf.ed.ac.uk/mfourman/teaching/mlCourse/notes/L01.pdf}\\
\textbf{Basics of Standard ML}\\
\copyright\ Michael P. Fourman\\
перевод \ref{fourmanru}

\bibitem{gentleml}
\url{http://www.soc.napier.ac.uk/course-notes/sml/manual.html}\\
\textbf{A Gentle Introduction to ML}\\
\copyright\ Andrew Cumming, Computer Studies, Napier University, Edinburgh

\bibitem{harper} \url{http://www.cs.cmu.edu/~rwh/smlbook/book.pdf}\\
\textbf{Programming in Standard ML}\\
\copyright\ Robert Harper, Carnegie Mellon University

\secly{Электроника и цифровая техника}

\bibitem{bcollis} \bibfig{../bib/bcollis.jpeg}\\
\textbf{An Introduction to Practical Electronics, Microcontrollers
and Software Design}\\
Bill Collis\\
2 edition, May 2014\\
\url{http://www.techideas.co.nz/}

\secly{Конструирование и технология}
\subsecly{Приемы ручной обработки материалов}
\subsecly{Механообработка}

\bibitem{tabletop} \bibfig{../bib/tabletop.jpg}\\
\textbf{Tabletop Machining}\\
Martin, Joe and Libuse, Craig\\
Sherline Products, 2000

\bibitem{briney}
\textbf{Home Machinists Handbook}\\
Briney, Doug, 2000

\bibitem{vasil}
\textbf{Маленькие станки}\\
Евгений Васильев\\
Псков, 2007\\
\url{http://www.coilgun.ru/stanki/index.htm}

\secly{Использование OpenSource программного обеспечения}

\subsecly{\LaTeX}

\bibitem{lvovsky}
\textbf{Набор и вёрстка в системе \LaTeX}\\
С.М.\,Львовский\\
3-е издание, исправленное и дополненное, 2003\\
\url{http://www.mccme.ru/free-books/llang/newllang.pdf}

\bibitem{ebooktex}
\textbf{e-Readers and \LaTeX}\\
Alan Wetmore\\
\url{https://www.tug.org/TUGboat/tb32-3/tb102wetmore.pdf}

\bibitem{isotex}
\textbf{How to cite a standard (ISO, etc.) in Bib\LaTeX\,?}\\
\url{http://tex.stackexchange.com/questions/65637/}

\subsecly{Математическое ПО: Maxima, Octave, GNUPLOT,..}

\bibitem{maxphis}
\textbf{Система аналитических вычислений Maxima для физиков-теоретиков}\\
В.А. Ильина, П.К.Силаев\\
\url{http://tex.bog.msu.ru/numtask/max07.ps} 

\subsecly{САПР, электроника, проектирование печатных плат}

\secly{Программирование}

\subsecly{GNU Toolchain}

\bibitem{bogo}
\textbf{Embedded Systems Programming in \cpp}\\
\copyright\ \url{http://www.bogotobogo.com/}\\
\url{http://www.bogotobogo.com/cplusplus/embeddedSystemsProgramming.php}

\bibitem{kumar}
\textbf{Embedded Programming with the GNU Toolchain}\\
Vijay Kumar B.\\
\url{http://bravegnu.org/gnu-eprog/}

\subsecly{Python}

\bibitem{py}
\textbf{Язык программирования Python}\\
Россум, Г., Дрейк, Ф.Л.Дж., Откидач, Д.С., Задка, М.,  Левис, М.,  Монтаро, С.,
Реймонд, Э.С., Кучлинг, А.М.,  Лембург, М.-А.,  Йи, К.-П.,  Ксиллаг, Д.,
Петрилли, Х.Г., Варсав, Б.А.,  Ахлстром, Дж.К.,  Роскинд, Дж.,  Шеменор, Н.,
Мулендер, С.\\
\copyright\ Stichting Mathematisch Centrum, 1990–1995
and Corporation for National Research Initiatives, 1995–2000
and BeOpen.com, 2000
and Откидач, Д.С., 2001 \\
\url{http://rus-linux.net/MyLDP/BOOKS/python.pdf}

Python является простым и, в то же время, мощным интерпретируемым
объектно-ориентированным языком программирования. Он предоставляет структуры
данных высокого уровня, имеет изящный синтаксис и использует динамический
контроль типов, что делает его идеальным языком для быстрого написания различных
приложений, работающих на большинстве распространенных платформ. Книга содержит
вводное руководство, которое может служить учебником для начинающих, и
справочный материал с подробным описанием грамматики языка, встроенных
возможностей и возможностей, предоставляемых модулями стандартной библиотеки.
Описание охватывает наиболее распространенные версии Python: от 1.5.2 до 2.0.

\subsecly{Разработка операционных систем и низкоуровневого ПО}

\bibitem{osdev} \textbf{OSDev Wiki}\\
\url{http://wiki.osdev.org}

\secly{Базовые науки}

\subsecly{Математика}

\bibitem{bermant} \bibfig{../bib/bermant.png}\\
\textbf{Краткий курс математического анализа для ВТУЗов}\\
Бермант А.Ф, Араманович И.Г.\\
М.: Наука, 1967\\
\url{https://drive.google.com/file/d/0B0u4WeMjO894U1Y1dEJ6cncxU28/view?usp=sharing}\\

Пятое издание известного учебника, охватывает большинство вопросов программы по
высшей математике для инженерно-технических специальностей вузов, в том числе
дифференциальное исчисление функций одной переменной и его применение к
исследованию функций; дифференциальное исчисление функций нескольких переменных;
интегральное исчисление; двойные, тройные и криволинейные интегралы; теорию
поля; дифференциальные уравнения; степенные ряды и ряды Фурье. Разобрано много
примеров и задач из различных разделов механики и физики. \emph{Отличается
крайней доходчивостью и отсутвием филонианов и ``легко догадаться''.}

\bibitem{varver}
\textbf{Математическая статистика}
Б.Л. Ван дер Варден

\bibitem{vanalg}
\textbf{Алгебра}
Б.Л. Ван дер Варден

\bibitem{kostalg}
\textbf{Введение в алгебру. В 3 частях. Часть 1. Основы алгебры}
А.И. Кострикин

\bibitem{kostlin}
\textbf{Введение в алгебру. В 3 частях. Линейная алгебра. Часть 2}
А.И. Кострикин

\bibitem{kremer} \bibfig{../bib/kremer.jpg}\\
\textbf{Теория вероятностей и математическая статистика}\\
Наум Кремер\\
М.: Юнити, 2010

\bibitem{smitdsp}\ \bibfig{../bib/smitdsp.jpg}\\
\textbf{Цифровая обработка сигналов. Практическое руководство для инженеров и
научных работников}\\
Стивен Смит\\
Додэка XXI, 2008\\ISBN 978-5-94120-145-7

В книге изложены основы теории цифровой обработки сигналов. Акцент сделан на
доступности изложения материала и объяснении методов и алгоритмов так, как они
понимаются при практическом использовании. Цель книги - практический подход к
цифровой обработке сигналов, позволяющий преодолеть барьер сложной математики и
абстрактной теории, характерных для традиционных учебников. Изложение материала
сопровождается большим количеством примеров, иллюстраций и текстов программ

\bibitem{data}
\textbf{Начала обработки экспериментальных данных}\\
Б.А.Князев, В.С.Черкасский\\
Новосибирский государственный университет, кафедра общей физики,
Новосибирск, 1996\\
\url{http://www.phys.nsu.ru/cherk/Metodizm_old.PDF}

Учебное пособие предназначено для студентов естественно-научных специальностей,
выполня- ющих лабораторные работы в учебных практикумах. Для его чтения
достаточно знаний математики в объеме средней школы, но оно может быть полезно и
тем, кто уже изучил математическую статистику, поскольку исходным моментом в нем
является не математика, а эксперимент. Во второй части пособия подробно описан
реальный эксперимент — от появления идеи и проблем постановки эксперимен- та до
получения результатов и обработки данных, что позволяет получить менее
формализованное представление о применении математической статистики. Посо- бие
дополнено обучающей программой, которая позволяет как углубить и уточнить
знания, полученные в методическом пособии, так и проводить собственно обработ-
ку результатов лабораторных работ. Приведен список литературы для желающих
углубить свои знания в области математической статистики и обработки данных.

\bibitem{richt}\ \bibfig{../bib/richt}\\
\textbf{Принципы современной математической физики}
Р. Рихтмайер

\bibitem{ts}
\textbf{Уравнения математической физики}
А.Н. Тихонов, А.А. Самарский

\bibitem{tszad}
\textbf{Сборник задач по математической физике}
Будак Б.М., Самарский А.А., Тихонов А.Н.

\bibitem{danko}\ \bibfig{../bib/danko}\\
\textbf{Высшая математика в упражнениях и задачах}\\
П.Е. Данко, А.Г.Попов, Т.Я. Кожевникова, С.П. Данко

\subsecly{Физика}

\bibitem{saveliev} \bibfig{../bib/saveliev.jpg}\\
Савельев И.В.

\bibfig{../bib/fein.jpg}
\textbf{Фейнмановские лекции по физике}\\
Ричард Фейнман, Роберт Лейтон, Мэттью Сэндс

\bibitem{fein1}
\href{https://drive.google.com/file/d/0B0u4WeMjO894SG9RdUtzZWVhQ1E/view?usp=sharing}{\textbf{Современная
наука о природе. Законы механики.}}

\bibitem{fein2}
\href{https://drive.google.com/file/d/0B0u4WeMjO894NjRqckpvM19hQVE/view?usp=sharing}{\textbf{Пространство. Время. Движение.}}

\bibitem{fein3} \href{}{\textbf{Излучение. Волны. Кванты.}}

\bibitem{fein4} \href{}{\textbf{Кинетика. Теплота. Звук.}}

\bibitem{fein5}
\href{https://drive.google.com/file/d/0B0u4WeMjO894dGdvM19ZUTh0UGM/view?usp=sharing}{\textbf{Электричество и магнетизм.}}

\bibitem{fein6} \href{}{\textbf{Электродинамика.}}

\bibitem{fein7} \href{}{\textbf{Физика сплошных сред.}}

\bibitem{fein8} \href{}{\textbf{Квантовая механика 1.}}

\bibitem{fein9} \href{}{\textbf{Квантовая механика 2.}}

\bibitem{kva}
\textbf{Основы квантовой механики}
Д.И. Блохинцев

\subsecly{Химия}

\bibitem{cirel}\ \bibfig{../bib/cirel}\\
\textbf{Квантовая химия. Молекулы, молекулярные системы и твердые тела. Учебное
пособие}
Владимир Цирельсон

\bibitem{calchem}\ \bibfig{../bib/calchem}\\
\textbf{Вычислительные методы для инженеров-химиков}
Х. Розенброк, С. Стори

\bibitem{shriver}\ \bibfig{../bib/shriver}\\
\textbf{Неорганическая химия} В 2 томах
Д. Шрайвер, П. Эткинс

\secly{Стандарты и ГОСТы}

\bibitem{gost2701}
2.701-2008 \textbf{Схемы. Виды и типы. Общие требования к выполнению}\\
\url{http://rtu.samgtu.ru/sites/rtu.samgtu.ru/files/GOST_ESKD_2.701-2008.pdf}

\end{thebibliography}
